\documentclass[11pt,a4paper]{scrartcl}

\usepackage[ngerman]{babel}
\usepackage[utf8]{inputenc}
\usepackage{amsmath}
\usepackage{mathtools}
\usepackage[hidelinks]{hyperref}

\def\phistar{\dot{\phi}^*}
\def\phii{\dot{\phi}_i}
\def\phiip{\dot{\phi}_{i+1}}
\def\thetai{\dot{\theta}_i(\theta)}
\def\pki{\frac{P}{K_i}}
\def\thetas{\frac{\dot{\theta}_{i+1}}{\dot{\theta}_i}}

\title{Herleitung Theorie}
\date{\today}
\author{Jeremiah Lübke}

\begin{document}

\maketitle

\section{Energie}
Kinetische Energie des i-ten Steins:
\[
    K_i = \phii^2 \frac{m}{6}L^2\sec(\phi)
\]
Verlust potentieller Energie nach einer Drehung um \(\theta\):
\[
    P = -m g L \sec(\phi) \left[ \cos(\phi) - \cos(\theta - \phi) \right]
\]
Man erhält:
\begin{equation}\label{eq:pk}
    \implies -\pki = \frac{2\omega^2}{\phii^2} \left[ \cos(\phi) -
    \cos(\theta-\phi) \right]
\end{equation}
wobei \(\omega^2 = \frac{3g\cos(\phi)}{2L}\)

\section{Geschwindigkeiten}
\emph{Annahme:} Geschwindigkeitsübertragung
\begin{equation}\label{eq:I}
    \frac{\phiip}{\dot{\psi}_i} = \frac{1+e}{R} \frac{k-1}{k} 
    = \frac{1}{R} \frac{k-1}{k}
\end{equation}
wobei inelastische Stöße \(\left( e = 0 \right)\) und Energieerhaltung der
Wellenfront \(\left( \frac{k-1}{k} \right)\) angenommen werden.
\(R = 1+\frac{\xi + \mu\lambda}{\xi - \mu h} \) ist der 
\emph{geometrische Koeffizient}. \\
\par\noindent
\emph{Annahme:} Verhältnis von Initial- und Kollisionsgeschwindigkeit
\begin{equation}
    \frac{\dot{\psi}_i}{\phii} = \sqrt{ \left( \frac{k}{k-1} \right)
    \left( 1-\frac{1}{k}\pki \right) }
    \label{eq:II}
\end{equation}
Der zweite Term in der Wurzel drückt den Energieverlust während des Fallens
aus.\\
\par\noindent
Man erhält damit das Verhältnis der Initialgeschwindigkeiten:
\begin{equation}\label{eq:III}
    \xRightarrow{\eqref{eq:I}} \phiip = \dot{\psi}_i \frac{1}{R} \frac{k-1}{k}
    \overset{\eqref{eq:II}}{=} \frac{\phii}{R} \sqrt{ \left( \frac{k-1}{k}
    \right) \left( 1- \frac{1}{k} \pki \right) }
\end{equation}
und wenn man den konstanten Kollisionswinkel \(\dot{\psi}_i\) zu dem variablen
Neigungswinkel \(\dot{\theta}_i\) überführt, erhält man:
\begin{equation}\label{eq:IV}
    \xRightarrow{\eqref{eq:II}} \dot{\theta}_i = \phii \sqrt{ \left( \frac{k}{k-1}
    \right) \left( 1- \frac{1}{k}\pki \right) }
\end{equation}
\par\noindent
Mit \eqref{eq:III} findet man die Initialgeschwindigkeit jedes Steins welche
zur Integration von \eqref{eq:IV} zum Berechnen der Transversalgeschwindigkeit
benötigt wird. \\
\par\noindent
Die Gleichung für die spezifische Geschwindigkeit erhält man, indem in
\eqref{eq:III} \[\phii = \phiip := \phistar\] setzt, \eqref{eq:pk} einsetzt
und dann nach dem noch in \eqref{eq:pk} enthaltenen \(\phistar\) umstellt.

\end{document}


