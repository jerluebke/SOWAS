\documentclass[11pt,a4paper]{scrartcl}

\usepackage[ngerman]{babel}
\usepackage[utf8]{inputenc}
\usepackage{amsfonts}
\usepackage{amsmath}
\usepackage{amssymb}
\usepackage{graphicx}
    \graphicspath{{img/}}
\usepackage[hidelinks]{hyperref}
\usepackage{lmodern}
\usepackage{fancyhdr}
    \pagestyle{fancy}
\usepackage{caption}
\usepackage[backend=bibtex, style=numeric-comp]{biblatex}
    \bibliography{Literatur.bib}
\usepackage{float}

\def\phistar{\dot{\phi}^*(\lambda)}
\def\phii{\dot{\phi}_i}
\def\phiip{\dot{\phi}_{i+1}}
\def\kl{\tilde{k}(\lambda)}
\def\klt{k(\theta, \lambda)}
\def\kt{\tilde{k}}
\def\kth{k(\theta)}
\def\thetastar{\dot{\theta}^*(\theta, \lambda)}
\def\thetai{\dot{\theta}_i(\theta)}
\def\pkstar{\left[\frac{P}{K}\right]^*(\theta, \dot{\phi}^*)}
\def\pki{\frac{P}{K_i}}
\def\pkith{\left[\frac{P}{K_i}\right](\theta)}
\def\thetas{\frac{\dot{\theta}_{i+1}}{\dot{\theta}_i}}

\title{Fallende Dominosteine}
\subtitle{Bearbeitungszeitraum: 09.04.2018 - 22.06.2018}
\subject{Theorie Zusammenfassung\\Gruppe F\\Ruhr-Universität Bochum}
\date{}
\author{}


\begin{document}
	%\fancyfoot[C]{\thepage}

	\maketitle
	\thispagestyle{empty}
	\newpage

	\pagenumbering{arabic}
	\setcounter{page}{1}

	\section{Wichtige Relationen aus Stronge (87) \label{Stronge87}}

Wir betrachten im Folgenden entkoppelte Dominosteine, dass heißt es berühren sich maximal zwei Steine gleichzeitig. Diese Annahme scheint nach der Betrachtung in \cite{Stronge87} sinnvoll, da nach der Kollision, der anstoßende Stein etwas zurückspringt und keinen Kontakt mehr mit dem nächsten Stein hat.

Die erste wichtige Gleichung, welche die initiale kinetische Energie der Steine beschreibt, ist gegeben durch die Energie der Rotation mit der gegebenen Geometrie des Problems \cite{Stronge87}: 

\begin{equation}
    \label{eq:K_0i}
	K_{0,initial} = \frac{1}{6 \cos^2\phi} M L^2 {\dot{\phi}_{initial}}^2 
\end{equation}

\begin{equation}
    \label{eq:frac_energy}
    \frac{\Delta P_i}{K_{0,i}} = \frac{2\omega^2}{{\dot{\phi}_{initial}}^2} (\cos\psi - \cos\phi)
\end{equation}

Wobei $ \omega^2 = \frac{3g\cos\phi}{2L} $ die Eigenfrequenz darstellt \cite{Stronge87}. Die letzte Gleichung beschreibt die relative Zunahme der Energie, hervorgerufen durch weitere Rotation um $ \psi $. Wir erhalten durch weitere Rechnungen:

\begin{equation}
	\label{eq:frac_freq}
	\frac{\dot{\psi}_{i}}{\dot{\phi}_{i}} = \sqrt{1- \frac{\Delta P_i}{K_{0,i}}}
\end{equation}

Wir wollen jetzt den Impulsübertrag auf den nächsten Stein, unter der Annahme dass $\mathrm{D}_n+1$ einen rückwirkenden Impuls auf $\mathrm{D}_n$ überträgt, betrachten. Im folgenden wird der Impuls mit $I$ bezeichnet. Es ist erforderlich in eine Komponente senkrecht auf die Kontaktfläche, sowie in eine Komponente entlang der Kontaktfläche zu unterteilen. Die Komponente parallel zur Fläche wird mit dem Reibungskoeffizienten $\mu$ gekennzeichnet. Die senkrechte Komponente ist durch den Parameter $\gamma = \frac{\xi + \mu\lambda}{L}$ gegeben. $\xi = L \cos(\psi + \phi)$ gibt an, in welcher Höhe der Stein $\mathrm{D_{n+1}}$ von $\mathrm{D}_{n}$ getroffen wird. Zusammenfassend gilt damit \cite{Stronge87}:
 
\begin{equation}
	\label{eq:parallel}
	I_{\parallel} = \mu I
\end{equation}
\begin{equation}
    \label{eq:senkrecht}
    I_{\perp} = \gamma I
\end{equation}

An dieser Stelle wird der Rückwirkungskoeffizient $e$ eingeführt, welcher den bereits erwähnten Rückstoß, also einen Energieverlust, beschreibt. $R = 1 + \frac{\xi + \lambda\mu}{\xi - \lambda\mu}$ ist ein geometrischer Parameter. Nach weiterer Rechnung erhält man:

\begin{equation}
    \label{eq:frac_init}
	\frac{\dot{\phi_{i+1}}}{\dot{\phi_{i}}} = \frac{1+e}{R} \sqrt{1 - \frac{\Delta P_i}{K_{0,i}}}
\end{equation}

Unter der Annahme, dass die Folge konvergiert \cite{Stronge87}, folgt für $ \phi_{i+1} = \phi_{i} $:

\begin{equation}
    \label{eq:finalfreq}
	\frac{\dot{\phi^{\star}}}{\omega} = (1+e)\sqrt{\frac{2(\cos\phi - \cos\psi)}{R^2 - (1+e)^2}}
\end{equation}

An dieser Stelle können wir diverse charakteristische Verhaltensweisen des Domino-Arrays feststellen. Bei der Betrachtung der Parameter $R$ und $e$, lässt sich die Beziehung $ R > 1+e $ aufstellen. Zusammen mit Gleichung (\ref{eq:finalfreq}) ergibt sich, dass die Geometrie des Problems eine weitaus größere Rolle spielt, als die Reibung. 

Desweiteren sei angemerkt, dass $\dot{\phi^{\star}}$ mit $\lambda$ steigt. Betrachten wir jedoch auch die Reibung, erhalten wir ein Maximum, nach dem die initialen Geschwindigkeiten wieder fallen.

Wir betrachten nun den Energieverlust gemäß:

\begin{equation}
    \label{eq:energyloss}
    \frac{\Delta K_i}{K_{0,i}} = (\frac{1-e}{1+e} (R-1)^2 + \frac{2}{1+e} (R-1) - 1)(\frac{\dot{\phi_{i+1}}}{\dot{\phi_{i}}})^2
\end{equation}

Wird der Array mit einer anderen Geschwindigkeit als $\dot{\phi^{\star}}$, $\dot{\phi_i} = \alpha \dot{\phi^{\star}}$ angestoßen, so erhalten wir mit Gleichung (\ref{eq:frac_freq}) eine Reskursionsformel, die die zeitliche Entwicklung beschreibt.

	\section{Wichtige Relationen aus Stronge / Shu (88) \label{Stronge88}}

Nach der Betrachtung in \cite{Stronge87}, wird jetzt angenommen, dass das Umwerfen der Dominos durch das Aneinanderlehnen der Steine zustande kommt. Auch hier wird die Existenz einer charakteristischen Geschwindigkeit vermutet. Die Annahmen sind wie folgt \cite{Stronge88}:

\begin{itemize}
	\item[(i)] Hinter der Wellenfront liegen unendlich  viele Dominosteine
	\item[(ii)] Jeder Stein lehnt gegen seinen direkten Nachbarn
	\item[(iii)] Die Reibung beim Rutschen über die Kontaktflächen ist vernachlässigbar klein
	\item[(iv)] Der Rückschlagkoeffizient ist vernachlässigbar, d.h. es gibt nur eine Kollision zwischen zwei Dominos.
\end{itemize}

Mit der Bedingung (ii) ergibt sich folgende trigonometrische Beziehung \cite{Stronge88}:

\begin{equation}
	\label{eq:trig}
	\sin(\Theta_{i+1} - \Theta_i) = \frac{\lambda + h}{L}\cos\Theta_i - \frac{h}{L}
\end{equation}

Damit erhalten wir folgende Rekursion für die Winkelgeschwindigkeiten:

\begin{equation}
    \label{eq:recursion}
	\frac{\dot{\Theta_{i+1}}}{\dot{\Theta_i}} = 1-(\frac{\lambda+h}{L})\frac{\sin\Theta_i}{\cos(\Theta_{i+1}-\Theta_i)}
\end{equation}

Damit hängen also alle Winkelgeschwindigkeiten lediglich von der Anstoßgeschwindigkeit des ersten Dominos ab. Sei also $ \dot{\phi_1} = \dot{\Theta_1}(0)$.

Zum Anfangszeitpunkt kann die Gesamtenergie $K$ beschrieben werden durch $K=k K_{0,i}$. Der Faktor $k$ wird berechnet durch \cite{Stronge88}:

\begin{equation}
    \label{eq:k}
    \begin{split}
        k & = \sum_{i=1}^{\infty} \frac{K_i}{K_{0,i}} \\
        & = 1+\sum_{j=1}^{\infty}(\prod_{i=1}^{j} (\frac{\dot{\Theta_{i+1}}}{\dot{\Theta_i}})^2)
    \end{split}
\end{equation}

Ähnlich zu den vorherigen Gleichungen erhalten wir eine Gleichung für das Verhältnis der potentiellen- zur kinetischen Energie, wobei $ \hat{\Theta} = \frac{1}{\arccos(1+\frac{\lambda}{h})} $ der maximal mögliche Winkel der Rotation ist.

\begin{equation}
    \label{eq:fracenergy2}
    -\frac{p}{K_1}=\frac{2\omega^2}{\dot{\phi_1}^2}(\cos\phi-\cos(\hat{\Theta}-\phi))
\end{equation}

Achtung: Es wird im folgenden Rückwärts gezählt. Das heißt $\mathrm{D}_1$ wäre der Domino an vordester Front.

Man erhält durch weitere Rechnung das Verhältnis von Kollisionsgeschwindigkeit und Initialgeschwindigkeit \cite{Stronge88}:

\begin{equation}
    \label{eq:fraccollision}
	\frac{\dot{\psi}_1}{\dot{\phi}_1} = \sqrt{(\frac{k}{k-1})(1-\frac{p}{kK_1})}
\end{equation}

Wenn $e=0$, folgt für den Energieverlust pro Kollision:

\begin{equation}
    \label{eq:energyloss2}
	\Delta K = K_1 \frac{\dot{\phi}_0 \dot{\psi}_1}{{\dot{\phi}_1}^2}
\end{equation}

\begin{equation}
	\label{eq:fracinitspeeds}
	\frac{\dot{\phi}_0}{\dot{\phi}_1} = \sqrt{(\frac{k-1}{k})(1-\frac{p}{kK_1})}
\end{equation}

Insgesamt erhalten wir nun für einen stationären Zustand, also eine gleichbleibende Geschwindigkeit: 

\begin{equation}
	\label{eq:phistar}
	\frac{\dot{\phi}^{\star}}{\omega} = \sqrt{2(\frac{k-1}{k})(\cos\phi-\cos(\hat{\Theta}-\phi))}
\end{equation}

Die durchschnittliche Geschwindigkeit der Wellenfront erhalten wir durch simple Integration \cite{Stronge88} gemäß:

\begin{equation}
    \label{eq:vel}
    v_{\star} = \frac{\lambda + h}{\int_0^{\psi} \frac{1}{\dot{\theta_1}(\theta)}\mathrm{d}x}
\end{equation}
\begin{equation}
    \label{eq:thetadot}
	\dot{\theta_1}(\theta) = \dot{\phi^{\star}} \sqrt{(\frac{k(\Theta)}{k(\Theta)-1})(1-\frac{p(\Theta)}{k(\Theta) K_{1}(\Theta)})}
\end{equation}


    \clearpage
	\section{Mathematische Vorgehensweise \label{MathVorg}}

\setcounter{equation}{0}
\renewcommand{\theequation}{\Roman{equation}}

\subsection{Was ist zu berechnen?}
\begin{enumerate}
    \item Spezifische Winkel- und Transversalgeschwindigkeit in Abhängigkeit
        des Abstandes und der Anzahl der in der Berechnung berücksichtigten
        Steine:
        $\dot{\phi}^*(\lambda, N), V^*(\lambda, N)$
    \item Verlauf von Winkel- und Transversalgeschwindigkeit in Abhängigkeit
        des Ortes für einen festen Abstand:
        $\phii = \dot{\phi}(x), V(x)$\\
        (mit $x = i(\lambda + h)$ für $i \in \{1, \cdots, \text{Anzahl
        der Steine}\}$)
\end{enumerate}
Dabei wird nur der Winkel $\theta_1$ des vordersten fallenden Steins benötigt,
welcher im Moment des Anstoßens \textit{Null} ist oder beim Integrieren der
Geschwindigkeiten durch die Integrationsschritte gegeben ist. Die Winkel der
anderen Steine können rekursiv mit \eqref{eq:geom} berechnet werden. Mit
diesen Werten kann dann alles weitere berechnet werden; nur den
Reibungskoeffizienten $\mu$ gilt es durch den Vergleich mit den experimentellen
Daten zu bestimmen.

\subsection{Gleichungen}
\subsubsection{Zentrale Gleichungen - Schritt 1}
\begin{subequations}
    Spezifische Winkelgeschwindigkeit:
    \begin{align}
    \phistar = \omega \sqrt{\left(
        \frac{\kl-1}{\kl} \right) \left( \frac{2\left[
    \cos(\phi)-\cos(\hat{\theta}(\lambda)-\phi)\right]}{\kl R^2 -\kl+1} \right)}
    \end{align}
\end{subequations}

\begin{subequations}
    Spezifische Transversalgeschwindigkeit:
    \begin{align}
        V^*(\lambda) &= (\lambda + h) / \int_0^\psi
        \frac{\mathrm{d}\theta}{\thetastar} \\
        \thetastar &= \phistar \sqrt{\left(
            \frac{\klt}{\klt-1} \right) \left(
        1-\frac{1}{\klt}\pkstar\right)}
    \end{align}
\end{subequations}

\subsubsection{Zentrale Gleichungen - Schritt 2}
\begin{subequations}
    Winkelgeschwindigkeit als Funktion des Ortes:
    \[
    \phiip = \frac{\phii}{R}\sqrt{\left(\frac{\kt-1}{\kt}\right)
    \left(1-\frac{1}{\kt}\pki\right)} \tag{IIIa}
    \]
\end{subequations}

\begin{subequations}
    Transversalgeschwindigkeit als Funktion des Ortes:
    \begin{align}
    V(x) &= (\lambda+h)/\int_0^\psi \frac{\mathrm{d}\theta}{\thetai}
    \\
    \thetai &= \phii \sqrt{\left(\frac{\kth}{\kth-1}\right)
    \left(1-\frac{1}{\kth}\pkith\right)}
    \end{align}
\end{subequations}

\subsubsection{Hilfsgleichungen}
\begin{subequations}
    Verhältnis potentieller und kinetischer Energie:
    \begin{align}
        -\pkith &= \frac{2\omega^2}{\phii^2} \left[\cos(\phi) -
        \cos(\theta-\phi)\right] \\
        -\pki &= -\left[\pki\right](\theta = \hat{\theta}) \nonumber \\
        -\pkstar &= -\left[\pki\right](\theta, \phii = \dot{\phi}^*) \nonumber
    \end{align}
\end{subequations}
\begin{subequations}
    Energieübertragungsrate:
    \begin{align}
        \klt &= \displaystyle\sum_{i=1}^N \frac{K_i}{K_1} \nonumber \\
        &= 1+\displaystyle\sum_{j=1}^N \displaystyle\prod_{i=1}^j
        \left(\thetas\right)^2 \\
        \kt &= k(\theta_1 = 0, \lambda) \nonumber
    \end{align}
\end{subequations}
\begin{subequations}\label{eq:geom}
    Geometrische Zusammenhänge:
    \begin{align}
        \thetas &= 1-\left(\frac{\lambda+h}{L}\right)
        \frac{\sin(\theta_i)}{\cos(\theta_{i+1}-\theta_i)} \\
        \sin(\theta_{i+1}-\theta_i) &= \left(\frac{\lambda+h}{L}\right)
        \cos(\theta_i)-\frac{h}{L}
    \end{align}
\end{subequations}
\vspace{\baselineskip}\\
Der geometrische Koeffizient R in Abhängigkeit des Reibungskoeffizienten
$\mu$ und der Auftreffhöhe $\xi$:
\[ R = 1+\frac{\xi + \mu\lambda}{\xi - \mu h} \]
Der minimale Neigungswinkel für einen instabilen Zustand $\phi$:
\[ \phi = \arctan(\frac{h}{L}) \]
Der Auftreffwinkel mit dem nächsten Stein:
\[ \psi = \arcsin(\frac{\lambda}{L}) \]
Der Ruhewinkel, wenn alle Steine gefallen sind:
\[ \hat{\theta} = \arccos(\frac{h}{h+\lambda}) \]

\subsection{Herleitungen}
\textit{TODO}

\subsection{Ausblick}
Wenn sich die hier vorgestellte Theorie als ergiebig erweist, können darauf
aufbauend weitere Aspekte des Versuchsaufbaus betrachtet werden, zum Beispiel
wie sich die Winkelgeschwindigkeiten bei gegebener Initialgeschwindigkeit an
den spezifischen Wert annähern.\\
Auch kann die Theorie dahingehend modifiziert werden, dass nicht mehr eine
Gruppe an Steinen betrachtet wird, sondern immer nur das gerade fallende Paar.
Bei so einem Ansatz kann auch ein eventuelles ``Zurückspringen'' der Steine berücksichtigt werden.


    \printbibliography

\end{document}
