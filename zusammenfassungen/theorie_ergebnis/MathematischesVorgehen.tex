\section{Mathematische Vorgehensweise \label{MathVorg}}

\setcounter{equation}{0}
\renewcommand{\theequation}{\Roman{equation}}

\subsection{Was ist zu berechnen?}
\begin{enumerate}
    \item Spezifische Winkel- und Transversalgeschwindigkeit in Abhängigkeit
        des Abstandes und der Anzahl der in der Berechnung berücksichtigten
        Steine:
        $\dot{\phi}^*(\lambda, N), V^*(\lambda, N)$
    \item Verlauf von Winkel- und Transversalgeschwindigkeit in Abhängigkeit
        des Ortes für einen festen Abstand:
        $\phii = \dot{\phi}(x), V(x)$\\
        (mit $x = i(\lambda + h)$ für $i \in \{1, \cdots, \text{Anzahl
        der Steine}\}$)
\end{enumerate}
Dabei wird nur der Winkel $\theta_1$ des vordersten fallenden Steins benötigt,
welcher im Moment des Anstoßens \textit{Null} ist oder beim Integrieren der
Geschwindigkeiten durch die Integrationsschritte gegeben ist. Die Winkel der
anderen Steine können rekursiv mit \eqref{eq:geom} berechnet werden. Mit
diesen Werten kann dann alles weitere berechnet werden; nur den
Reibungskoeffizienten $\mu$ gilt es durch den Vergleich mit den experimentellen
Daten zu bestimmen.

\subsection{Gleichungen}
\subsubsection{Zentrale Gleichungen - Schritt 1}
\begin{subequations}
    Spezifische Winkelgeschwindigkeit:
    \begin{align}
    \phistar = \omega \sqrt{\left(
        \frac{\kl-1}{\kl} \right) \left( \frac{2\left[
    \cos(\phi)-\cos(\hat{\theta}(\lambda)-\phi)\right]}{\kl R^2 -\kl+1} \right)}
    \end{align}
\end{subequations}

\begin{subequations}
    Spezifische Transversalgeschwindigkeit:
    \begin{align}
        V^*(\lambda) &= (\lambda + h) / \int_0^\psi
        \frac{\mathrm{d}\theta}{\thetastar} \\
        \thetastar &= \phistar \sqrt{\left(
            \frac{\klt}{\klt-1} \right) \left(
        1-\frac{1}{\klt}\pkstar\right)}
    \end{align}
\end{subequations}

\subsubsection{Zentrale Gleichungen - Schritt 2}
\begin{subequations}
    Winkelgeschwindigkeit als Funktion des Ortes:
    \[
    \phiip = \frac{\phii}{R}\sqrt{\left(\frac{\kt-1}{\kt}\right)
    \left(1-\frac{1}{\kt}\pki\right)} \tag{IIIa}
    \]
\end{subequations}

\begin{subequations}
    Transversalgeschwindigkeit als Funktion des Ortes:
    \begin{align}
    V(x) &= (\lambda+h)/\int_0^\psi \frac{\mathrm{d}\theta}{\thetai}
    \\
    \thetai &= \phii \sqrt{\left(\frac{\kth}{\kth-1}\right)
    \left(1-\frac{1}{\kth}\pkith\right)}
    \end{align}
\end{subequations}

\subsubsection{Hilfsgleichungen}
\begin{subequations}
    Verhältnis potentieller und kinetischer Energie:
    \begin{align}
        -\pkith &= \frac{2\omega^2}{\phii^2} \left[\cos(\phi) -
        \cos(\theta-\phi)\right] \\
        -\pki &= -\left[\pki\right](\theta = \hat{\theta}) \nonumber \\
        -\pkstar &= -\left[\pki\right](\theta, \phii = \dot{\phi}^*) \nonumber
    \end{align}
\end{subequations}
\begin{subequations}
    Energieübertragungsrate:
    \begin{align}
        \klt &= \displaystyle\sum_{i=1}^N \frac{K_i}{K_1} \nonumber \\
        &= 1+\displaystyle\sum_{j=1}^N \displaystyle\prod_{i=1}^j
        \left(\thetas\right)^2 \\
        \kt &= k(\theta_1 = 0, \lambda) \nonumber
    \end{align}
\end{subequations}
\begin{subequations}\label{eq:geom}
    Geometrische Zusammenhänge:
    \begin{align}
        \thetas &= 1-\left(\frac{\lambda+h}{L}\right)
        \frac{\sin(\theta_i)}{\cos(\theta_{i+1}-\theta_i)} \\
        \sin(\theta_{i+1}-\theta_i) &= \left(\frac{\lambda+h}{L}\right)
        \cos(\theta_i)-\frac{h}{L}
    \end{align}
\end{subequations}
\vspace{\baselineskip}\\
Der geometrische Koeffizient R in Abhängigkeit des Reibungskoeffizienten
$\mu$ und der Auftreffhöhe $\xi$:
\[ R = 1+\frac{\xi + \mu\lambda}{\xi - \mu h} \]
Der minimale Neigungswinkel für einen instabilen Zustand $\phi$:
\[ \phi = \arctan(\frac{h}{L}) \]
Der Auftreffwinkel mit dem nächsten Stein:
\[ \psi = \arcsin(\frac{\lambda}{L}) \]
Der Ruhewinkel, wenn alle Steine gefallen sind:
\[ \hat{\theta} = \arccos(\frac{h}{h+\lambda}) \]

\subsection{Herleitungen}
\textit{TODO}

\subsection{Ausblick}
Wenn sich die hier vorgestellte Theorie als ergiebig erweist, können darauf
aufbauend weitere Aspekte des Versuchsaufbaus betrachtet werden, zum Beispiel
wie sich die Winkelgeschwindigkeiten bei gegebener Initialgeschwindigkeit an
den spezifischen Wert annähern.\\
Auch kann die Theorie dahingehend modifiziert werden, dass nicht mehr eine
Gruppe an Steinen betrachtet wird, sondern immer nur das gerade fallende Paar.
Bei so einem Ansatz kann auch ein eventuelles ``Zurückspringen'' der Steine berücksichtigt werden.
