\section{Wichtige Relationen aus Stronge (87) \label{Stronge87}}

Wir betrachten im Folgenden entkoppelte Dominosteine, dass heißt es berühren sich maximal zwei Steine gleichzeitig. Diese Annahme scheint nach der Betrachtung in \cite{Stronge87} sinnvoll, da nach der Kollision, der anstoßende Stein etwas zurückspringt und keinen Kontakt mehr mit dem nächsten Stein hat.

Die erste wichtige Gleichung, welche die initiale kinetische Energie der Steine beschreibt, ist gegeben durch die Energie der Rotation mit der gegebenen Geometrie des Problems \cite{Stronge87}: 

\begin{equation}
    \label{eq:K_0i}
	K_{0,initial} = \frac{1}{6 \cos^2\phi} M L^2 {\dot{\phi}_{initial}}^2 
\end{equation}

\begin{equation}
    \label{eq:frac_energy}
    \frac{\Delta P_i}{K_{0,i}} = \frac{2\omega^2}{{\dot{\phi}_{initial}}^2} (\cos\psi - \cos\phi)
\end{equation}

Wobei $ \omega^2 = \frac{3g\cos\phi}{2L} $ die Eigenfrequenz darstellt \cite{Stronge87}. Die letzte Gleichung beschreibt die relative Zunahme der Energie, hervorgerufen durch weitere Rotation um $ \psi $. Wir erhalten durch weitere Rechnungen:

\begin{equation}
	\label{eq:frac_freq}
	\frac{\dot{\psi}_{i}}{\dot{\phi}_{i}} = \sqrt{1- \frac{\Delta P_i}{K_{0,i}}}
\end{equation}

Wir wollen jetzt den Impulsübertrag auf den nächsten Stein, unter der Annahme dass $\mathrm{D}_n+1$ einen rückwirkenden Impuls auf $\mathrm{D}_n$ überträgt, betrachten. Im folgenden wird der Impuls mit $I$ bezeichnet. Es ist erforderlich in eine Komponente senkrecht auf die Kontaktfläche, sowie in eine Komponente entlang der Kontaktfläche zu unterteilen. Die Komponente parallel zur Fläche wird mit dem Reibungskoeffizienten $\mu$ gekennzeichnet. Die senkrechte Komponente ist durch den Parameter $\gamma = \frac{\xi + \mu\lambda}{L}$ gegeben. $\xi = L \cos(\psi + \phi)$ gibt an, in welcher Höhe der Stein $\mathrm{D_{n+1}}$ von $\mathrm{D}_{n}$ getroffen wird. Zusammenfassend gilt damit \cite{Stronge87}:
 
\begin{equation}
	\label{eq:parallel}
	I_{\parallel} = \mu I
\end{equation}
\begin{equation}
    \label{eq:senkrecht}
    I_{\perp} = \gamma I
\end{equation}

An dieser Stelle wird der Rückwirkungskoeffizient $e$ eingeführt, welcher den bereits erwähnten Rückstoß, also einen Energieverlust, beschreibt. $R = 1 + \frac{\xi + \lambda\mu}{\xi - \lambda\mu}$ ist ein geometrischer Parameter. Nach weiterer Rechnung erhält man:

\begin{equation}
    \label{eq:frac_init}
	\frac{\dot{\phi_{i+1}}}{\dot{\phi_{i}}} = \frac{1+e}{R} \sqrt{1 - \frac{\Delta P_i}{K_{0,i}}}
\end{equation}

Unter der Annahme, dass die Folge konvergiert \cite{Stronge87}, folgt für $ \phi_{i+1} = \phi_{i} $:

\begin{equation}
    \label{eq:finalfreq}
	\frac{\dot{\phi^{\star}}}{\omega} = (1+e)\sqrt{\frac{2(\cos\phi - \cos\psi)}{R^2 - (1+e)^2}}
\end{equation}

An dieser Stelle können wir diverse charakteristische Verhaltensweisen des Domino-Arrays feststellen. Bei der Betrachtung der Parameter $R$ und $e$, lässt sich die Beziehung $ R > 1+e $ aufstellen. Zusammen mit Gleichung (\ref{eq:finalfreq}) ergibt sich, dass die Geometrie des Problems eine weitaus größere Rolle spielt, als die Reibung. 

Desweiteren sei angemerkt, dass $\dot{\phi^{\star}}$ mit $\lambda$ steigt. Betrachten wir jedoch auch die Reibung, erhalten wir ein Maximum, nach dem die initialen Geschwindigkeiten wieder fallen.

Wir betrachten nun den Energieverlust gemäß:

\begin{equation}
    \label{eq:energyloss}
    \frac{\Delta K_i}{K_{0,i}} = (\frac{1-e}{1+e} (R-1)^2 + \frac{2}{1+e} (R-1) - 1)(\frac{\dot{\phi_{i+1}}}{\dot{\phi_{i}}})^2
\end{equation}

Wird der Array mit einer anderen Geschwindigkeit als $\dot{\phi^{\star}}$, $\dot{\phi_i} = \alpha \dot{\phi^{\star}}$ angestoßen, so erhalten wir mit Gleichung (\ref{eq:frac_freq}) eine Reskursionsformel, die die zeitliche Entwicklung beschreibt.
