\section{Einleitung}
Die Gruppe möchte im folgenden Semester den Einsatz von Simulationen in der
Physik erlernen, sowie die Grenzen einer solchen austesten. Als Beispiel dient
hier eine Kette von fallenden Dominosteinen. Die physikalische Beschreibung des
Systems kann – je nach Art der Betrachtung – beliebig komplex werden.

Analytische Lösungen lieferten bereits vergangene Paper und Studentenprojekte.
Diese beschränken sich jedoch auf die einfachste Art der Beschreibung. Die
Gruppe sieht genau hier große Verbesserungsmöglichkeiten!

Das kommende Projekt soll daher mehr über die Dynamik des Systems herausfinden:
\\
Wie steht die Geschwindigkeit einer Dominokette im Zusammenhang mit dem Abstand
der Steine, bzw. der Anstoßgeschwindigkeit? Welche Rolle spielt die
Anstoßgeschwindigkeit des ersten Steins, stellt sich immer die gleiche
Endgeschwindigkeit ein? Wie groß sind die Reibungseinflüsse, wovon hängen sie
ab? Was genau passiert beim Übergang von Haft- zu Gleitreibung?
