\section{Experimentelle und theoretische Vorgehensweise}
Das Projekt wird sich ungefähr gleich aus einem theoretischen, sowie einem
experimentellen Teil zusammensetzen.

\subsection{Theoretischer Teil 1}
Im ersten Teil der Durchführung soll eine Simulation erstellt werden, in der
Dominosteine mit unendlicher Haftreibung betrachtet werden. Dabei werden die
physikalischen Grundlagen genutzt, welche vergangene analytische Betrachtungen
bereits lieferten (Siehe dazu Literaturliste, \ref{Grundlagen} Physikalische
Grundlagen).

Das aufgestellte Modell soll am Ende Antworten auf die folgenden Fragen geben
können:
Wie verhält sich die Geschwindigkeit der Dominokette bei Variation des Abstandes
zwischen den Steinen? Erreicht das System immer denselben stationären Zustand,
egal mit welcher Anstoßgeschwindigkeit der erste Stein fällt?

Ziel dieser Vorgehensweise ist es, zu überprüfen ob die Simulation überhaupt
eine physikalische Gültigkeit besitzt. Dazu wird an dieser Stelle das erste
Experiment durchgeführt.

\subsection{Experimenteller Teil 1}
Auf einer gummierten Unterlage – diese garantiert einen hohen
Haftreibungskoeffizienten – werden Dominosteine in gleichmäßigen Abständen
aufgebaut. Die Länge derselben steht zu diesem Zeitpunkt noch nicht fest, da
noch nicht bekannt ist, wie lange das System braucht um eine konstante
Geschwindigkeit zu erreichen.
Mithilfe einer selbstgebauten Vorrichtung, wird nun der erste Stein angestoßen
und anschließend die Geschwindigkeit der Dominos mithilfe von zwei
Lichtschranken bestimmt.

Es werden zwei Messreihen durchgeführt:
In der ersten Reihe wird ein typischer Abstand für einen solchen Aufbau
gewählt und dabei die Anstoßgeschwindigkeit variiert. Im Anschluss wird dann
die Anstoßgeschwindigkeit konstant gehalten und der Abstand variiert.

\subsection{Theoretischer Teil 2}
Ergibt die Auswertung der erhobenen Daten, dass das aufgestellte Modell präzise
genug ist, kann selbiges erweitert werden. Die Dominosteine rotieren jetzt nicht
mehr um einen festen Punkt. Die Energie der Steine setzt sich zusammen aus einer
Translations- sowie Rotationsbewegung. Zusätzlich wird daher der Übergang von
Haft- zu Gleitreibung betrachtet.
Die Erweiterung der Simulation liefert nun einen Aufschluss darüber, wie sich
die Steine auf anderen Unterlagen Verhalten.

\subsection{Experimenteller Teil 2}
Im letzten Teil der Durchführung werden Experimente, ähnlich wie zuvor,
durchgeführt. Der Versuchsaufbau ändert sich grundsätzlich nicht, jedoch werden
nun auch die Unterlagen ausgetauscht. Es erscheint sinnvoll hier von sehr
glatten, bis zu sehr rauen Oberflächen alles auszuprobieren. Dieses Vorgehen
ermöglicht später den Gültigkeitsbereich der gewonnenen Erkenntnisse anzugeben.

In der Auswertungsphase sollen nun die Ergebnisse diskutiert werden. Hier kann
entweder angegeben werden, warum die Simulation keine realitätsnahen Ergebnisse
liefert, beziehungsweise, inwiefern die Simulation noch weiter verbessert
werden kann.
