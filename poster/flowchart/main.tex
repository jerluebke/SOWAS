\documentclass[11pt,a4paper]{scrartcl}

\usepackage[ngerman]{babel}
\usepackage[utf8]{inputenc}
\usepackage{amsfonts}
\usepackage{amsmath}
\usepackage{amssymb}
\usepackage{tikz}
\usetikzlibrary{shapes.geometric,arrows}
\usetikzlibrary{graphs}

\everymath{\displaystyle}

\def\phii{\dot{\phi}_i}
\def\phiip{\dot{\phi}_{i+1}}
\def\thetai{\theta_i}
\def\thetaim{\theta_{i-1}}
\def\thetaid{\dot{\theta}_i}
\def\thetainit{\dot{\theta}_i^{Initial}}
\def\thetainitm{\dot{\theta}_{i-1}^{Initial}}
\def\thetastoss{\dot{\theta}_{i-1}^{Stoß}}
\def\thetadstar{\dot{\theta}_*}

\newcommand{\funcof}[1]{
    f\left(#1\right)
}

\newcommand{\start}{
    Koordinaten des i-ten Dominos: $\thetai, \thetaid$\\
    Aus der Geometrie: $\thetaim = \funcof{\thetai}$
}
\newcommand{\dyn}{
    Dynamik des Problems liefert:
    $\thetaid = f\left(\thetainit, \theta\right)$
}
\newcommand{\enconv}{
    Die Energie zwischen zwei Stößen ist erhalten:
    $\thetainit = f\left(\thetastoss\right)$
}
\newcommand{\ang}{
    Eine rekursive Folge für alle Dominos:
    $\thetainit = f\left(\thetainitm\right)$
}
\newcommand{\vi}{$
    v_i = \frac{d+w}{t_i},
    t_i = \int_0^{\theta^{Stoß}}
    \frac{\mathrm{d}\theta}{\thetaid\left(\theta\right)}
$}
\newcommand{\limang}{
    $\lim_{i \to \infty} \thetaid = \thetadstar$
}
\newcommand{\res}{
    $v_* = \frac{d+w}{t_*}$
}

\tikzstyle{start} = [rectangle, rounded corners, minimum width=3cm, minimum
height=1cm, text centered, text width=4cm, draw=black, fill=yellow!30]
\tikzstyle{proc} = [rectangle, rounded corners, minimum width=3cm, minimum
height=1cm, text centered, text width=4cm, draw=black, fill=blue!30]
\tikzstyle{result} = [rectangle, rounded corners, minimum width=3cm, minimum
height=1cm, text centered, draw=black, fill=red!30]
\tikzstyle{coord} = [circle, inner sep=0pt]


\begin{document}

\begin{tikzpicture}[>=stealth, thick, black!50, text=black,
                    every new ->/.style={shorten >=1pt},
                    graphs/every graph/.style={edges=rounded corners},
                    vh path/.style={to path={-| (\tikztotarget)}},
                    hv path/.style={to path={|- (\tikztotarget)}}]
    \matrix [row sep=5mm, column sep=1cm] {
        \node (start) [start]  {\start}; & \\
        \node (dyn)   [proc]   {\dyn};   & \node (ene) [proc]  {\enconv}; \\
        \node (p2)    [coord]  {};       & \node (p3)  [coord] {};        \\
        \node (vi)    [result] {\vi};    & \node (ang) [proc]  {\ang};    \\
        \node (p4)    [coord]  {};       & \node (lim) [proc]  {\limang}; \\
        \node (res)   [result] {\res}; \\
    };
    \graph [use existing nodes] {
        start -> [vh path] ene -> ang -> lim -> p4;
        start -> dyn -> vi -> res;
        dyn -> [hv path] p3;
    };
\end{tikzpicture}

\end{document}
