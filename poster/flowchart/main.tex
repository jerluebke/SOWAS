\documentclass[11pt,a4paper]{scrartcl}

\usepackage[ngerman]{babel}
\usepackage[utf8]{inputenc}
\usepackage{amsfonts}
\usepackage{amsmath}
\usepackage{amssymb}
\usepackage{tikz}
\usetikzlibrary{shapes.geometric,arrows}

\def\phii{\dot{\phi}_i}
\def\phiip{\dot{\phi}_{i+1}}
\def\thetai{\dot{\theta}_i(\theta)}

\tikzstyle{start} = [rectangle, rounded corners, minimum width=3cm, minimum
height=1cm, text centered, draw=black, fill=yellow!30]
\tikzstyle{proc} = [rectangle, rounded corners, minimum width=3cm, minimum
height=1cm, text centered, draw=black, fill=blue!30]
\tikzstyle{result} = [rectangle, rounded corners, minimum width=3cm, minimum
height=1cm, text centered, draw=black, fill=red!30]


\begin{document}

\begin{tikzpicture}[node distance=3cm]
    \node (start) [start] {Koordinaten};
    \node (dyn) [proc, below of=start] {Dynamik};
    \node (enconv) [proc, right of=dyn] {Energieerhaltung};
    \node (tran) [result, below of=dyn] {$V_i$};
    \node (ang) [proc, right of=tran] {$\dot{\theta}_i$};
    \node (conv) [proc, below of=ang] {Grenzwert};
    \node (res) [result, below left of=conv] {$V_*$};
    
\end{tikzpicture}
    
\end{document}
