\documentclass[11pt,a4paper]{scrartcl}

\usepackage[ngerman]{babel}
\usepackage[utf8]{inputenc}
\usepackage{amsfonts}
\usepackage{amsmath}
\usepackage{amssymb}
\usepackage{tikz}
\usetikzlibrary{shapes.geometric,arrows}

\def\phii{\dot{\phi}_i}
\def\phiip{\dot{\phi}_{i+1}}
\def\thetai{\dot{\theta}_i(\theta)}

\def\start{Koordinaten}
\def\dyn{Dynamik}
\def\enconv{Energieerhaltung}
\def\vi{$v_i$}
\def\ang{$\dot{\theta}_i$}
\def\lim{Grenzwert}
\def\res{$v_*$}

\tikzstyle{start} = [rectangle, rounded corners, minimum width=3cm, minimum
height=1cm, text centered, draw=black, fill=yellow!30]
\tikzstyle{proc} = [rectangle, rounded corners, minimum width=3cm, minimum
height=1cm, text centered, draw=black, fill=blue!30]
\tikzstyle{result} = [rectangle, rounded corners, minimum width=3cm, minimum
height=1cm, text centered, draw=black, fill=red!30]
\tikzstyle{coord} = [circle, inner sep=0pt, minimum size=2pt, fill=red]


\begin{document}

\begin{tikzpicture}
    \matrix[row sep=5mm, column sep=1cm]{
        \node (start) [start]  {\start}; & \node (p1)  [coord] {};        \\
        \node (dyn)   [proc]   {\dyn};   & \node (ene) [proc]  {\enconv}; \\
        \node (p2)    [coord]  {};       & \node (p3)  [coord] {};        \\
        \node (vi)    [result] {\vi};    & \node (ang) [proc]  {\ang};    \\
        \node (p3)    [coord]  {};       & \node (lim) [proc]  {\lim};    \\
        \node (res)   [result] {\res}; \\
    };
\end{tikzpicture}

\end{document}
